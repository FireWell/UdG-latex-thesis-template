%% This is the `thesispreamble.tex' file,

%% Phd Thesis Template
%% ===================
%% 
%% Based  on the template of University of Wollongong
%%   2013-07-16, Thomas Griffiths, tmg994(at)uowmail.edu.au
%% Modified by 
%%   2014, Miguel Hernando Camelo
%%   2015, Ferney Maldonado, Jaime Chavarriaga

%----------------------------------------------------------------------------%
% I encourage you to read the documentation for each of the packages below,  %
% they contain instructions for implementation and examples of their use.    %
% If you get stuck read my comments, hopefully they can help you find where  %
% your answers will be. I highly reccomend making friends with Mr. Google,   %
% he knows quite a bit. The wikibook on LaTeX is also very helpful:          %
% http://en.wikibooks.org/wiki/LaTeX/                                        %
% The packages I've loaded here are the bare basics. and mainly deal with    %
% formatting, captions and things. There are thousands of packages out there %
% for all the disciplines and formatting you might need. Google what you're  %
% looking for and the keywords 'LaTeX' and 'package', You'll probably find   %
% what you're looking for. I encourage you to look on ctan, www.ctan.org,    %
% for packages that might be relevant to your degree. If you're in science I % 
% can recomend the siunitx and the chemmachro (or mhchem) package. They make %
% it really easy to typeset chemical equations, any quantity with units and  %
% scientific notation.                                                       %
%                                                                            %
%----------------------------------------------------------------------------%

\usepackage{geometry}
	\geometry{a4paper,inner=3cm, outer=3cm, top=3cm, bottom=3cm}
	% Dimensions from UOW thesis guidelines.
	\pdfpagewidth=\paperwidth 
	\pdfpageheight=\paperheight
	% This acts as a failsafe to ensure things aren't stretched or moved when it's finally printed as a PDF.

%\usepackage[parfill]{parskip} 
% Activate to begin paragraphs with an empty (return) line, comment out the indent below if you chose the return line option.

\setlength{\parindent}{4ex}	% Sets the length of the paragraph indent. Current setup has a an indent. Disable this if you activate the return line above.
	
\usepackage{setspace}
% Double or one and a half spacing.
	
\usepackage{graphicx}
	\DeclareGraphicsRule{.tif}{png}{.png}{`convert #1 `dirname #1`/`basename #1 .tif`.png}
% Graphics. Remove me and you won't have any figures, and that would be very boring.

\usepackage[usenames,dvipsnames,svgnames,table]{xcolor}
% Adds the ability to make coloured text and lines throughout the document. See documentation for xcolor.

%-------------------- Tables, figures and captions
\usepackage[font={small},labelfont={bf},margin=4ex]{caption}
% Makes bold labeled and smaller font captions. Must be loaded before the longtable package to avoid conflicts! 

\usepackage{longtable} 
% Long tables (more than one page). Different headers and footers for beginning and end pages, etc.

\usepackage{afterpage} 
% Make a longtable start on the next clear page, but fills the previous one with text first (no random gaps in the text-from long tables anymore! Man, the day I discovered this...)

\usepackage{booktabs} 
% Nice looking tables and lines in tables

\usepackage{multirow} 
% Entries in tables over multiple rows

\usepackage{lscape} 
% Pages in landscape

\usepackage{pdflscape} 
% Landscape pages also rotated in the pdf

\usepackage{wrapfig} 
% Allows figures that don't take up the entire width of the page, wrapping the text around the figure

%\usepackage[position=top,singlelinecheck=false,captionskip=4pt]{subfig} 
% Multiple figures in an individual figure. Fig. 1 a, b, c, etc. each with, or without, it's own individual caption, and with a global caption for all sub figures.

%-------------------- Special symbols and fonts
\usepackage{amssymb} 
% Maths symbols

%-------------------- Document sections, headers, footers, and bibliography

% footnotes at page bottom
\usepackage[bottom]{footmisc}
\usepackage{fancyhdr}											
% for creating different headers and footers

%-------------------- Bibliography
%\usepackage[backend=biber,articletitle=true,style=chem-rsc,doi=false]{biblatex}
%\usepackage[backend=biber,style=numeric-comp, sorting=none, clearlang=true, sortcites=true, doi=false, url=false]{biblatex}
% This is the package that lets you create a bibliography. I recommend reading the biblatex documentation to understand all the options i've specified here. BibLaTeX was created to replace BibTeX. It has lots of extra fields and options. I'm also using the biber backend here rather than the default, it copes with unicode and so can deal with accented characters easily.

% Currently this is set up to use RSC style references with article titles displayed.

% Traditionally you would use BibTeX, a special build of TeX, the newer biblatex package is a more powerful bibliograpy management tool for LaTeX. You can make multiple chapter based bibliographies, footnote bibliographes, sort your references by date, author, order cited, essentially by any bit of citation data you happen to have. You can also have a seperate library with a differnet format for say books and articles. Or if you're a PhD student, the thesis references and your publications.

%\usepackage[numbers,super,comma,sort&compress]{natbib}				
	%\setcitestyle{square}										
	% places citations in square brackets to helps to distinguish between powers and citations
%This is the old natbib package that meshes with bibtex (rather than using the newer biblatex). It's here mainly for legacy purposes. Try to shift to biblatex if you can, it is cleaner in it's implementation and creating a custom citation style is easier then with bibtex.

\usepackage[unicode=true,colorlinks=true,linkcolor=black,citecolor=black,urlcolor=black,breaklinks=true]{hyperref}
% The hyperref package allows you to have clickable links in your pdf. It also allows you to have the mailto link associated with your name. It can be  a bit finicky, so load it last.

%-------------------- Command renewals, New commands etc.
\renewcommand{\thefootnote}{\alph{footnote}}							
%letters for footnotes instead of numbers to avoid confusion with references.
