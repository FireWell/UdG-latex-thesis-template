%% This is the `thesispreamble.tex' file,

%% Phd Thesis Template
%% ===================
%% 
%% Based  on the template of University of Wollongong
%%   2013-07-16, Thomas Griffiths, tmg994(at)uowmail.edu.au
%% Modified by 
%%   2014, Miguel Hernando Camelo
%%   2015, Ferney Maldonado, Jaime Chavarriaga

%----------------------------------------------------------------------------%
% I encourage you to read the documentation for each of the packages below,  %
% they contain instructions for implementation and examples of their use.    %
% If you get stuck read my comments, hopefully they can help you find where  %
% your answers will be. I highly reccomend making friends with Mr. Google,   %
% he knows quite a bit. The wikibook on LaTeX is also very helpful:          %
% http://en.wikibooks.org/wiki/LaTeX/                                        %
% The packages I've loaded here are the bare basics. and mainly deal with    %
% formatting, captions and things. There are thousands of packages out there %
% for all the disciplines and formatting you might need. Google what you're  %
% looking for and the keywords 'LaTeX' and 'package', You'll probably find   %
% what you're looking for. I encourage you to look on ctan, www.ctan.org,    %
% for packages that might be relevant to your degree. If you're in science I % 
% can recomend the siunitx and the chemmachro (or mhchem) package. They make %
% it really easy to typeset chemical equations, any quantity with units and  %
% scientific notation.                                                       %
%                                                                            %
%----------------------------------------------------------------------------%

%% Page size
%% =========

\usepackage{geometry}
	\geometry{a4paper,inner=3cm, outer=3cm, top=3cm, bottom=3cm}
	% Dimensions from UOW thesis guidelines.
	\pdfpagewidth=\paperwidth 
	\pdfpageheight=\paperheight

%% Paragraph beginning 
%% ===================
%% -- use only one of them

%\usepackage[parfill]{parskip}		% begin paragraphs with an empty (return) line 
\setlength{\parindent}{4ex}			% sets the length of the paragraph indent.
	
\usepackage{setspace}				% Double or one and a half spacing.

%% Graphic Support
%% ===============

% use color names 
\usepackage[usenames,dvipsnames,svgnames,table]{xcolor}
	
\usepackage{graphicx}				% standard LaTeX graphics tool
	\DeclareGraphicsRule{.tif}{png}{.png}{`convert #1 `dirname #1`/`basename #1 .tif`.png}
\usepackage{subfigure}				% Multiples subfigures in a figure

%\usepackage[position=top,singlelinecheck=false,captionskip=4pt]{subfig}

\usepackage{pdfpages}       		% allows pdf inclusion
\usepackage{epstopdf}       		% used to accept eps format images. 

\usepackage{wrapfig} 				% Wrap texts around the figures


%% Tables, figures and captions
%% ============================

% Makes bold labeled and smaller font captions.
% Must be loaded before the longtable package to avoid conflicts!
\usepackage[font={small},labelfont={bf},margin=4ex]{caption}
  
% Long tables (more than one page). 
% Different headers and footers for beginning and end pages, etc.
\usepackage{longtable} 
% Make a longtable start on the next clear page
\usepackage{afterpage} 

% Nice looking tables and lines in tables
\usepackage{booktabs} 
% Entries in tables over multiple rows
\usepackage{multirow} 

\usepackage{array}						% enhanced support for tables


%% Landscape support
%% =================

\usepackage{lscape}				% Pages in landscape 
\usepackage{pdflscape} 			% Landscape pages also rotated in the pdf


%% Special symbols and fonts
%% =========================

\usepackage{amsfonts}				% includes math fonts
\usepackage{amsmath,amssymb,amsthm}	% Maths symbols		
\usepackage{mathptmx}       		% sets Times Roman as basic font
\usepackage{helvet}         		% sets Helvetica as sans-serif font
\usepackage{courier}        		% sets Courier as typewriter font
\usepackage{type1cm}        		% for font management

%% Document sections, headers and footers
%% ======================================

\usepackage[bottom]{footmisc}		% footnotes at page bottom
\usepackage{fancyhdr}				% for creating headers and footers

%% Bibliography
%% ============

\usepackage[backend=bibtex,maxnames=4, style=numeric-comp, sorting=none, clearlang=true, sortcites=true, doi=false, url=false, firstinits]{biblatex}

%\usepackage[backend=biber,articletitle=true,style=chem-rsc,doi=false]{biblatex}
%\usepackage[backend=biber,style=numeric-comp, sorting=none, clearlang=true, sortcites=true, doi=false, url=false]{biblatex}
 
% Backends:
% - bitext is the standard manager (it works on ShareLatex)
% - biber copes with unicode and so can deal with accented characters easily.

% clickable links in the PDF
\usepackage[unicode=true,colorlinks=true,linkcolor=black,citecolor=black,urlcolor=black,breaklinks=true]{hyperref}


%% Command renewals, New commands etc.
%% ===================================

% Use letters for footnotes instead of numbers (avoid confusion with references).
\renewcommand{\thefootnote}{\alph{footnote}}							

% new column types
\newcolumntype{P}[1]{>{\centering\arraybackslash}p{#1}}
\newcolumntype{M}[1]{>{\centering\arraybackslash}m{#1}}

% Special caption for figures that includes the source
% Usage:
%   \captionsource{Caption}{source}
%	\captionsource[Caption-in-TOC]{Caption}{source}
\newcommand*{\captionsource}[3][]{%
  \ifthenelse{\equal{#1}{}}%
  {%
  	\caption[{#2}]{%
    	#2%
    	\\\hspace{\linewidth}%
    	\textbf{Source:} #3%
	}%
  }{%
  	\caption[{#1}]{%
    	#2%
    	\\\hspace{\linewidth}%
    	\textbf{Source:} #3%
	}%
  }%
}

